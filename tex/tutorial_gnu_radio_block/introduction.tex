\section*{Introduction}

GNU radio uses software to command various block, allowing to rapidly construct and test communication blocks. It can be interesting to be able to create your own blocks and to dynamically update parameters of a block from another one. For example, the LimeSDR parameters, such as the gain, can be modified from GNU radio. It is however not trivial and we will will decompose this into two steps. The first one is to dynamically update a GNU radio variable from a given block. The second one is to automatically get the updated value of this variable in another block. In this tutorial, you will therefore learn the following manipulations:

\begin{itemize}
    \item How to create a new block in GNU radio.
    \item How to update dynamically a GNU radio variable.
    \item How to get the updated value of a variable automatically.
\end{itemize}

Finally, we will quickly discuss some important points regarding the update of LimeSDR parameters from GNU Radio.
