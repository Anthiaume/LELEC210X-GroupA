
\section{Demo D2a: power consumption}

During this LELEC2102 project, we expect you to be able to present quick demos on specific topics, as scheduled in the program for this semester. They can be seen as small checkpoints to make sure you master the important basic blocks of the project. These demos should take you less time than writing a detailed report while still providing a good occasion to develop new skills, learn by doing and finally get some feedback along the way! There is no need to write anything for a demo but you must make sure it will run "live" smoothly. \\

The \textbf{demo D2a} will focus on the power consumption of your MCU. Indeed, you programmed it and learned how to implement some ideas in practice: it is now time to see how much power it draws and if it works as expected. It is critical for embedded programming as you will need to fit into a power budget. For this demo, we expect you to:

\begin{enumerate}
    \item Show what is the impact of a LED blinking on the power consumption trace. Is it significant for embedded computing applications? Therefore, what is a good practice for battery-powered nodes?
    \item Observe the power consumption with and without \texttt{WFI()}. What is the difference? Why is it so?
    \item Observe how the power consumption is changing if we change the clock frequency. What could cause this important change?
\end{enumerate}
