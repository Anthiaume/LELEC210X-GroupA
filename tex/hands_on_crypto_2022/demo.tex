
\section{Demo}

We expect from you a working and verified implementation.
To validate your implementation, we expect you to generate \emph{test vectors}
and validate your test vectors against the given python implementation.
Your test vectors should cover many \texttt{app\_data} length (e.g. all lengths
from 0 to 1000) and various values for all the parameters (including the key).

\begin{bclogo}[couleur = gray!20, arrondi = 0.2, logo=\bcinfo]{Test vector testing tips}
In order to be efficient in your testing, do not perform the tests manually! Here are some tips:
\begin{itemize}
    \item In order to test the authentication, you don't need to run the full
        application: you may simply hard-code a set of messages that you send
        from the \texttt{main} function.
    \item You may generate random data on the MCU, but you may also (which
        makes your test results easier to reproduce) write a script that
        generates a C file that contains the test vectors.
    \item Keep using the UART to send packets and modify the given python
        script to check every incoming packet, signal only the failing test
        cases and report test statistics (e.g. how many test passed and failed,
        for which message lengths, etc.).
    \item Don't hesitate to use many test vectors!
\end{itemize}
\end{bclogo}
